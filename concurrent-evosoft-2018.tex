\documentclass[sigconf]{acmart}
%%% Local Variables:
%%% ispell-local-dictionary: "english"
%%% End:

\usepackage{booktabs} % For formal tables
\usepackage{graphicx}
\usepackage{rotating}
\definecolor{Gray}{gray}{0.6}

\setcopyright{rightsretained}

% DOI
\acmDOI{10.1145/nnnnnnn.nnnnnnn}

% ISBN
\acmISBN{978-x-xxxx-xxxx-x/YY/MM}


%Conference
\acmConference[GECCO '18]{the Genetic and Evolutionary Computation
Conference 2018}{July 15--19, 2018}{Kyoto, Japan}
\acmYear{2018}
\copyrightyear{2018}


\acmArticle{4}
\acmPrice{15.00}

\begin{document}

\title{Mapping evolutionary algorithms to a reactive, stateless architecture
  using a modern concurrent language}

\author{Juan J. Merelo}
\orcid{1234-5678-9012}
\affiliation{%
  \institution{Universidad de Granada}
  \streetaddress{Daniel Saucedo Aranda, s/n}
  \city{Granada}
  \country{Spain}
}
\email{jmerelo@ugr.es}

\author{Mario García-Valcez}
\affiliation{%
  \institution{Anonymous institute 2}
  \streetaddress{P.O. Box 1212}
  \city{Dublin}
  \state{Ohio}
  \postcode{43017-6221}
}
\email{2aa@ai2.com}

% The default list of authors is too long for headers.
\renewcommand{\shortauthors}{J. J. Merelo et al.}

\maketitle


\begin{abstract}
\keywords{concurrent programming, distributed evolutionary algorithms}
\end{abstract}

\begin{CCSXML}
<ccs2012>
<concept>
<concept_id>10003752.10003809.10003716.10011136.10011797.10011799</concept_id>
<concept_desc>Theory of computation~Evolutionary algorithms</concept_desc>
<concept_significance>500</concept_significance>
</concept>

<concept>
<concept_id>10010520.10010521.10010537.10003100</concept_id>
<concept_desc>Computer systems organization~Cloud computing</concept_desc>
<concept_significance>500</concept_significance>
</concept>

<concept>
<concept_id>10010147.10010919.10010172</concept_id>
<concept_desc>Computing methodologies~Distributed algorithms</concept_desc>
<concept_significance>300</concept_significance>
</concept>
</ccs2012>
\end{CCSXML}

\ccsdesc[500]{Theory of computation~Evolutionary algorithms}

\ccsdesc[300]{Computing methodologies~Distributed algorithms}

\keywords{Microservices, distributed computing,
  event-based systems, Kappa architecture, stateless algorithms,
  algorithm implementation, performance evaluation, distributed
  computing, pool-based systems, heterogeneous distributed systems,
  serverless computing, functions as a service.}

\maketitle

\section{Introduction}

\noindent Genetic algorithms (GA) \cite{GA_Goldberg89} are currently
one of the most used meta-heuristics to solve engineering
problems. Furthermore, parallel genetic algorithms (pGAs) are useful
to find  solutions of complex optimizations problems in adequate times
\cite{Luque2011}; in particular, problems with complex fitness. Some
authors \cite{Alba2001} state that using pGAs improves the quality of
solutions in terms of the number of evaluations needed to find
one. This reason, together with the improvement in evaluation time
brought by the simultaneous running in several nodes, have made
parallel and distributed evolutionary algorithms a popular
methodology. 

Running evolutionary algorithms in parallel is quite straightforward, but programming paradigms used for the implementation of such algorithms is far from being an object of study. Object oriented or procedural languages like Java and C/C++ are mostly used. Even when some researchers show that implementation matters \cite{DBLP:conf/iwann/MereloRACML11}, parallels approaches in new languages/paradigms is not normally seen as a land for scientific improvements.

New parallel platforms have been identified as new trends in pGAs \cite{Luque2011}, however only hardware is considered. Software platforms, specifically programming languages, remain poorly explored; only Ada \cite{Santos2002} and Erlang \cite{A.Bienz2011,Kerdprasop2013} were slightly tested.

%The multicore’s challenge \cite{SutterL05} shows a current need for making parallel even the simplest program. But this way leads us to use and create design patterns for parallel algorithms; the conversion of a pattern into a language feature is a common practice in the programming languages domain, and sometimes that means a language modification, others the creation of a new one.

This work explores the advantages of some non mainstream languages (not included in the top ten of any most popular languages ranking) with concurrent and functional features in order to develop GAs in its parallel versions. It is motivated by the lack of community attention on the subject and the belief that using concepts that simplify the modeling and implementation of such algorithms might promote their use in research  and in practice.

This research is intended to show some possible areas of improvement on architecture and engineering best practices for concurrent-functional paradigms, as was made for Object Oriented Programming languages \cite{EO:FEA2000}, by focusing on pGAs as a domain of application and describing how their principal traits can be modeled by means of concurrent-functional languages constructs. We are continuing the research reported in \cite{DBLP:conf/gecco/CruzGGC13,J.Albert-Cruz2013}.
%other papers (hidden for double-blind review).

The rest of the paper is organized as follows. Next section presents the state of the art in concurrent and functional programming language paradigms and its potential use for implementing pGAs. Our proposal to adapt pGAs to the paradigms using a study case is explained in section \ref{sec:impl} as well as the experimental results in section \ref{sec:results}. In section \ref{sec:sample} we show a sample program of canonical island/GA implemented in Scala.

Finally, we draw the conclusions and future lines of work in section \ref{sec:conclusions}.


\section{State of the Art}
\noindent Developing correct software quickly and efficiently is a
never ending goal in the software industry. Novel solutions that try
to make a difference providing new abstraction tools outside the
mainstream of programming languages have been proposed to pursue this
goal; two of the most promising are the functional and the concurrent.

The concurrent programming paradigm (or concurrent oriented programming \cite{Armstrong2003}) is characterized by the presence of programming constructs for managing processes like first class objects. That is, with operators for acting upon them and the possibility of using them like parameters or function's result values. This simplifies the coding of concurrent algorithms due to the direct mapping between patterns of communications and processes with language expressions.

Concurrent programming is hard for many reasons, the communication/synchronization between processes is key in the design of such algorithms. One of the best efforts to formalize and simplify that is the Hoare’s {\em Communicating Sequential Processes} \cite{Hoare:1978:CSP:359576.359585}, this interaction description language is the theoretical support for many libraries and new programming languages.

When a concurrent programming language is used normally it has a
particular way of handling units of execution, being independent of
the operation system has several advantages: one program in those
languages will work the same way on different operating systems. Also
they can efficiently manage a lot of processes even on a
mono-processor machine.


Functional programming paradigm, despite its advantages, does not have
many followers. Several years ago was used in Genetic Programming
\cite{Briggs:2008:FGP:1375341.1375345,Huelsbergen:1996:TSE:1595536.1595579,walsh:1999:AFSFESIHLP}
and recently in neuroevolution \cite{Sher2013} but in GA its presence
is practically nonexistent \cite{Hawkins:2001:GFG:872017.872197}. 

This paradigm is characterized by the use of functions like first
class concepts, and for discouraging the use of state changes. The
latter is particularly useful for develop concurrent algorithms in
which the communication by state changes is the origin of errors and
complexity. Also, functional features like closures and first class
functions in general, allow to express in one expression patterns like
\emph{observer} which in language like Java need so many lines and
files of source code.


The field of programming languages research is very active in the
Computer Science discipline. To find software construction tools with
new and better means of algorithms expression is welcome. In the last
few years the functional and concurrent paradigms have produced a rich
mix in which concepts of the first one had been simplified by the use
of the second ones. 

Among this new generation, the languages Erlang and Scala have
embraced the actor model of concurrency and get excelentes results in
many application domains; Clojure is another one with concurrent
features such as promises/futures, Software Transaction Memory and
agents. All of these tools have processes like built-in types and
scale beyond the restrictions of the number of OS-threads.

\section{Conclusions}
\label{sec:conclusions}

\begin{acks}

  The author would like to acknowledge the support of grants\\
  taking\\
  this\\
  much\\
  space\\

\end{acks}


\bibliographystyle{ACM-Reference-Format}
\bibliography{geneura,referencias,erlang-ae-model}

\end{document}
